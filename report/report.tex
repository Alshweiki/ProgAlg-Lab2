\documentclass[11pt, oneside]{report}   	% use "amsart" instead of "article" for AMSLaTeX format
\usepackage[latin1]{inputenc} 
\usepackage[T1]{fontenc}
\usepackage{geometry}                		% See geometry.pdf to learn the layout options. There are lots.
\geometry{letterpaper}                   		% ... or a4paper or a5paper or ... 
%\geometry{landscape}                		% Activate for rotated page geometry
%\usepackage[parfill]{parskip}    		% Activate to begin paragraphs with an empty line rather than an indent
\usepackage{graphicx}				% Use pdf, png, jpg, or eps§ with pdflatex; use eps in DVI mode
								% TeX will automatically convert eps --> pdf in pdflatex		
\usepackage{amssymb}
\usepackage{listings}
\usepackage{caption}
\usepackage{hyperref}
 
\lstset{language=C,label=DescriptiveLabel}


\usepackage{titling}
\newcommand{\subtitle}[1]{
  \posttitle{
    \par\end{center}
    \begin{center}\large#1\end{center}
    \vskip0.5em}
}
\DeclareCaptionFormat{listing}{\rule{\dimexpr\textwidth+17pt\relax}{0.4pt}\vskip1pt#1#2#3}

\title{Distributed HPC programming with MPI/OpenMP}
\subtitle{Distributed Matrix Multiplication}
\author{Alshweiki Mhd Ali \thanks{\ mhdali.alshweiki@master.hes-so.ch}\\ 
Gugger Jo�l \thanks{\ joel.gugger@master.hes-so.ch}\\ 
Marguet Steve-David \thanks{\ stevedavid.marguet@master.hes-so.ch} \\ \\ user: ggroup20@grid11}
\date{\today}

\begin{document}
\maketitle


\pagenumbering{gobble}



\begin{abstract}
The objective of this exercise is to parallelize a square matrix multiplication program with MPI and OpenMP. This program computes the following product:
\[
	A \times B = R
\]
where A, B and R are NxN square matrices. The tested sizes N are: 1900, 2800, 3700, 4600, 5500, 6400, 7300, 8100, 9000, 9900.
\end{abstract}


\pagenumbering{arabic}

\chapter{Code MPI \& OpenMP}
La premi�re partie de notre travail � �t� d'implementer le parall�lisme en distribuant la charge de travail sur diff�rentes machines. Nous avons choisi d'utiliser 9 hosts. Le premier �tant le master, et les 8 autres des workers. La matrice est donc s�par�e en 8 parties.

En calculant le pourcentage du temps mesur� en parall�le par rapport au temps s�quentiel, on constate que pour toutes les valleures mesur�es nous arrivons en dessous des 7\% du temps s�quentiel. On constate �galement que le programme parall�le est de plus en plus performant lorsque la matrice grandi.

\[
\begin{tabular}{|r|r|r|}
\hline
Sequential time & Parallel time & Percentage \\
\hline
60.832	& 4.00	& 6.58\% \\
203.151	& 13.00	& 6.40\% \\
543.5105	& 28.00	& 5.15\% \\
1033.095	& 52.00	& 5.03\% \\
1872.805	& 84.74	& 4.52\% \\
5315.51	& 336.36	& 6.33\% \\
4481.75	& 183.97	& 4.10\% \\
6363.86	& 251.53	& 3.95\% \\
8717.87	& 352.56	& 4.04\% \\
12371.8	& 462.97	& 3.74\% \\
\hline
\end{tabular}
\]

\newpage



\begin{abstract}
\begin{center}
Les sources du projet sont disponibles sur GitHub � l'adresse suivante : \\
\href{https://github.com/Alshweiki/ProgAlg-Lab2}{https://github.com/Alshweiki/ProgAlg-Lab2}
\end{center}
\end{abstract}


\end{document}  














