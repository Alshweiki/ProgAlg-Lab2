\documentclass[11pt, oneside]{report}   	% use "amsart" instead of "article" for AMSLaTeX format
\usepackage[latin1]{inputenc} 
\usepackage[T1]{fontenc}
\usepackage{geometry}                		% See geometry.pdf to learn the layout options. There are lots.
\geometry{letterpaper}                   		% ... or a4paper or a5paper or ... 
%\geometry{landscape}                		% Activate for rotated page geometry
%\usepackage[parfill]{parskip}    		% Activate to begin paragraphs with an empty line rather than an indent
\usepackage{graphicx}				% Use pdf, png, jpg, or eps§ with pdflatex; use eps in DVI mode
								% TeX will automatically convert eps --> pdf in pdflatex		
\usepackage{amssymb}
\usepackage{listings}
\usepackage{caption}
\usepackage{hyperref}
 
\lstset{language=C,label=DescriptiveLabel}


\usepackage{titling}
\newcommand{\subtitle}[1]{
  \posttitle{
    \par\end{center}
    \begin{center}\large#1\end{center}
    \vskip0.5em}
}
\DeclareCaptionFormat{listing}{\rule{\dimexpr\textwidth+17pt\relax}{0.4pt}\vskip1pt#1#2#3}

\title{Matrix Multiplication with POP}
\subtitle{Performance analysis of a distributed matrix multiplication program}
\author{Alshweiki Mhd Ali \thanks{\ mhdali.alshweiki@master.hes-so.ch}\\ 
Gugger Jo�l \thanks{\ joel.gugger@master.hes-so.ch}\\ 
Marguet Steve-David \thanks{\ stevedavid.marguet@master.hes-so.ch} \\ \\ user: ggroup20@grid11}
\date{\today}

\begin{document}
\maketitle


\pagenumbering{gobble}



\begin{abstract}

The objective of this lab is to execute and to analyse the performances of a parallel square matrices multiplication program written in POP-C++ and in POP-Java. As for the MPI/OpenMP lab, these programs computes square matrices multiplication, i.e. the product $A \times B = R$ where $A$, $B$ and $R$ are $N \times N$ matrices (square matrix).

The program uses a � Master/Worker � approach. The master prepares the matrices, creates the workers (POP-C++ or POP-Java parallel objects), sends the work to do to each workers, waits for the partial result of each worker and finally reconstructs the $R$ matrix.

The algorithm behaves similarly to the one of the MPI/OpenMP lab by dividing the matrix $A$ in several bloc of lines and the matrix $B$ in several blocs of columns.

\end{abstract}


\pagenumbering{arabic}

\chapter{Computation of sequential references times}
The sequential reference time is the time used to do the computation using only one worker and one core.



\chapter{Computation of parallel times}
Each group will have to compute for five different sizes of the matrix ($N$), the time for five different numbers of workers ($W$).

Matrix sizes:
\[
\begin{tabular}{|r|}
\hline
Matrix sizes ($N$) \\
\hline
1080	\\
2160	\\
3240	\\
4620	\\
6240	\\
\hline
\end{tabular}
\]

Workers distribution:
\[
\begin{tabular}{|c|r|}
\hline
Workers ($W$) & $= LxC$ \\
\hline
2	& $=1x2$\\
4	& $=2x2$ \\
6	& $=2x3$ \\
9	& $=3x3$ \\
10	& $=5x2$ \\
\hline
\end{tabular}
\]

\newpage



\begin{abstract}
\begin{center}
Les sources du projet sont disponibles sur GitHub � l'adresse suivante : \\
\href{https://github.com/Alshweiki/ProgAlg-Lab2}{https://github.com/Alshweiki/ProgAlg-Lab2}
\end{center}
\end{abstract}


\end{document}  














